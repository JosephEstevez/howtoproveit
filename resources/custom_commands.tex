% ============================================================================
% CUSTOM COMMANDS FOR MATHEMATICAL PROOFS
% Enhanced version for "How to Prove It" exercises
% ============================================================================

% ----------------------------------------------------------------------------
% THEOREM ENVIRONMENTS
% ----------------------------------------------------------------------------
\newtheorem{theorem}{Theorem}[section]
\newtheorem{lemma}[theorem]{Lemma}
\newtheorem{definition}[theorem]{Definition}
\newtheorem{corollary}[theorem]{Corollary}
\newtheorem{proposition}[theorem]{Proposition}
\newtheorem{example}[theorem]{Example}
\newtheorem{remark}[theorem]{Remark}
\newtheorem{claim}[theorem]{Claim}

% Proof environment (if not using amsthm)
\newenvironment{solution}{\begin{proof}[Solution]}{\end{proof}}

% ----------------------------------------------------------------------------
% COMMON MATHEMATICAL SETS
% ----------------------------------------------------------------------------
\newcommand{\N}{\mathbb{N}}          % Natural numbers
\newcommand{\Z}{\mathbb{Z}}          % Integers
\newcommand{\Q}{\mathbb{Q}}          % Rational numbers
\newcommand{\R}{\mathbb{R}}          % Real numbers
\newcommand{\C}{\mathbb{C}}          % Complex numbers
\newcommand{\powerset}{\mathcal{P}}  % Power set
\newcommand{\emptySet}{\emptyset}    % Empty set

% ----------------------------------------------------------------------------
% SET OPERATIONS
% ----------------------------------------------------------------------------
\newcommand{\union}{\cup}            % Union
\newcommand{\intersection}{\cap}     % Intersection
\newcommand{\setdiff}{\setminus}     % Set difference
\newcommand{\symdiff}{\triangle}     % Symmetric difference
\newcommand{\Sub}{\subseteq}         % Subset (inclusive)
\newcommand{\Sup}{\supseteq}         % Superset (inclusive)
\newcommand{\propersubset}{\subsetneq} % Proper subset
\newcommand{\propersupset}{\supsetneq} % Proper superset

% ----------------------------------------------------------------------------
% LOGICAL OPERATORS AND SYMBOLS
% ----------------------------------------------------------------------------
\newcommand{\st}{\text{ such that }} % "such that"
\newcommand{\then}{\Rightarrow}      % Implication (legacy)
\newcommand{\imp}{\Rightarrow}       % Implies
\newcommand{\biconditional}{\Leftrightarrow} % If and only if
\newcommand{\contradiction}{\Rightarrow\Leftarrow} % Contradiction symbol
\newcommand{\NOT}{\neg}              % Logical not
\newcommand{\AND}{\wedge}            % Logical and
\newcommand{\OR}{\vee}               % Logical or

% ----------------------------------------------------------------------------
% INTERVALS
% ----------------------------------------------------------------------------
\newcommand{\openint}[2]{(#1, #2)}     % Open interval
\newcommand{\closedint}[2]{[#1, #2]}   % Closed interval
\newcommand{\halfopen}[2]{[#1, #2)}    % Half-open interval [a,b)
\newcommand{\halfclosed}[2]{(#1, #2]}  % Half-closed interval (a,b]

% ----------------------------------------------------------------------------
% FUNCTIONS AND RELATIONS
% ----------------------------------------------------------------------------
\newcommand{\domain}{\text{dom}}     % Domain
\newcommand{\range}{\text{ran}}      % Range
\newcommand{\image}{\text{im}}       % Image
\newcommand{\restrict}[2]{{#1}|_{#2}} % Function restriction
\newcommand{\compose}{\circ}         % Function composition
\newcommand{\id}{\text{id}}          % Identity function

% ----------------------------------------------------------------------------
% PROOF SHORTCUTS
% ----------------------------------------------------------------------------
\newcommand{\QED}{\hfill\blacksquare}       % QED symbol
\newcommand{\WLOG}{\text{WLOG}}             % Without loss of generality
\newcommand{\ie}{\text{i.e.}}              % That is
\newcommand{\eg}{\text{e.g.}}              % For example
\newcommand{\hence}{\text{Hence}}          % Hence
\newcommand{\thus}{\text{Thus}}            % Thus

% ----------------------------------------------------------------------------
% QUANTIFIERS (alternative symbols)
% ----------------------------------------------------------------------------
\newcommand{\forsome}{\exists}       % There exists
\newcommand{\All}{\forall}           % For all (explicit for clarity)
\newcommand{\unique}{\exists!}       % There exists unique

% ----------------------------------------------------------------------------
% COMMON ABBREVIATIONS
% ----------------------------------------------------------------------------
\newcommand{\wrt}{\text{w.r.t.}}     % With respect to
\newcommand{\ste}{\text{s.t.}}       % Such that (short form)
\newcommand{\resp}{\text{resp.}}     % Respectively

% ----------------------------------------------------------------------------
% SPECIAL FORMATTING
% ----------------------------------------------------------------------------
\newcommand{\highlight}[1]{\textbf{#1}}           % Highlight important text
\newcommand{\vocab}[1]{\textit{#1}}               % Vocabulary terms
\newcommand{\case}[1]{\textbf{Case #1:}}          % Case labels in proofs