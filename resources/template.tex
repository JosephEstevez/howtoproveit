\documentclass[11pt]{article}

% ============================================================================
% PACKAGE IMPORTS
% ============================================================================
\usepackage[utf8]{inputenc}
\usepackage[T1]{fontenc}
\usepackage{lmodern}

% Mathematics packages
\usepackage{amsmath, amssymb, amsthm}
\usepackage{mathtools}

% Page layout and formatting
\usepackage{geometry}
\geometry{margin=1in}
\usepackage{enumitem}
\usepackage{parskip}

% References and links
\usepackage{hyperref}
\hypersetup{
    colorlinks=true,
    linkcolor=blue,
    filecolor=magenta,      
    urlcolor=cyan,
    citecolor=red
}
\usepackage{cleveref}

% Headers and footers
\usepackage{fancyhdr}
\pagestyle{fancy}
\fancyhf{}
\fancyhead[L]{How to Prove It - Exercises}
\fancyhead[R]{\thepage}
\renewcommand{\headrulewidth}{0.4pt}

% Load custom commands (adjust path as needed)
% Theorem Environments
\newtheorem{theorem}{Theorem}[section]
\newtheorem{lemma}[theorem]{Lemma}
\newtheorem{definition}[theorem]{Definition}
\newtheorem{corollary}[theorem]{Corollary}
\newtheorem{proposition}[theorem]{Proposition}
\newtheorem{example}[theorem]{Example}

% Common Sets
\newcommand{\N}{\mathbb{N}}
\newcommand{\Z}{\mathbb{Z}}
\newcommand{\Q}{\mathbb{Q}}
\newcommand{\R}{\mathbb{R}}
\newcommand{\C}{\mathbb{C}}
\newcommand{\powerset}{\mathcal{P}}

% Shortcuts
\newcommand{\st}{\text{ such that }}
\newcommand{\then}{\Rightarrow}


% ============================================================================
% DOCUMENT INFORMATION
% ============================================================================
\title{Chapter X -- [Chapter Title]}
\author{[Your Name]}
\date{\today}

% ============================================================================
% DOCUMENT CONTENT
% ============================================================================
\begin{document}

\maketitle
\thispagestyle{empty} % Remove header from title page

% Optional: Table of contents for longer chapters
% \tableofcontents
% \newpage

% ============================================================================
% EXERCISE SECTIONS
% ============================================================================

\section*{Exercise X.X}
\addcontentsline{toc}{section}{Exercise X.X} % Add to TOC even though unnumbered

% Example structure for different types of proofs:

% For direct proofs:
% \begin{proof}
%     [Write your formal proof here.]
% \end{proof}

% For proof by contradiction:
% \begin{proof}[Proof by Contradiction]
%     Assume for the sake of contradiction that [negation of what we want to prove].
%     [Derive contradiction]
%     This is a contradiction. \contradiction
%     Therefore, [original statement] must be true.
% \end{proof}

% For proof by cases:
% \begin{proof}
%     We consider two cases:
%     
%     \case{1} [First case condition]
%     [Proof for first case]
%     
%     \case{2} [Second case condition] 
%     [Proof for second case]
%     
%     In both cases, [conclusion]. Therefore, [original statement] is true.
% \end{proof}

% Example exercise solution:
\begin{solution}
    Write your formal proof here using the custom scommands defined in \texttt{custom\_commands.tex}.
    
    For example: Let $A, B \subset \R$ such that $A \intersection B = \emptySet$. 
    We want to prove that $A \union B = A \cup B \setminus (A \intersection B)$.
    
    Since $A \intersection B = \emptySet$, we have $A \cup B \setminus (A \intersection B) = A \cup B \setminus \emptySet = A \cup B$.
    
    Therefore, $A \union B = A \cup B$ as desired. \qed
\end{solution}

% ============================================================================
% ADDITIONAL EXERCISES
% ============================================================================

\section*{Exercise X.Y}
\addcontentsline{toc}{section}{Exercise X.Y}

\begin{solution}
    [Another proof here]
\end{solution}

% ============================================================================
% REFERENCES (if needed)
% ============================================================================
% \begin{thebibliography}{9}
% \bibitem{velleman}
% Velleman, Daniel J. \textit{How to Prove It: A Structured Approach}. 
% Cambridge University Press, 2nd edition, 2006.
% \end{thebibliography}

\end{document}